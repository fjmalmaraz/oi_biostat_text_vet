\section{Exercises}
\label{exercisesChapterData}
\begin{problem}
Classify the following variables depending on their types (nominal,
ordinal, discrete, continuous)
\begin{multicols}{2}
  \begin{enumerate} \renewcommand{\theenumi}{\alph{enumi}}
\item Size of a T--Shirt (XS,S,M,L,XL,XXL);
\item Shoes size;
\item Credit card number;
\item Patient temperature;
\item Number of children in a household;
\item Last seen film;
\item Phone number;
\item Number of passed subjects of the first semester;
\item Cell phone company;
\item  Passport number;
\item The amount of money in a bank account;
\item The height of a building;
\item Animal weight; 
\item Horse breed; 
\item Body temperature;
\item Day of the week when an analysis is  performed; 
\item Number of puppies born in a delivery;
\item Brand of animal feed;
\item Whether or not a dog had a previous bone fracture.   
\end{enumerate}
\end{multicols}

\end{problem}



\begin{problem}
  The number of animals in a veterinary clinic   during 32 days are:
\begin{verbatim}
53 54 42 43 41 43 49 48 37 57 63 44 52 50 43 45
59 40 41 54 63 57 46 51 68 46 42 42 44 48 52 53
\end{verbatim}

\begin{enumerate} \renewcommand{\theenumi}{\alph{enumi}}
\item  Calculate 10--percentile, 35--percentile and 62--percentile.
\item Calculate all the quartiles.
\item If there are outlying values, locate it.

\item Draw a box plot.
\end{enumerate}
\end{problem}

\begin{problem}
  In a questionnaire about overweighted cats, the owner is asked for
  the number of times that the pet weight has been measured for the
  last 6 months. The answers  were: 

  \begin{center}
    \begin{tabular}{*{20}{c}}
      3& 5 & 2 & 0 & 2 & 1 & 6 & 2 & 0 & 6 & 2 & 0 & 4 & 3 & 3 & 5 & 2
      & 0 & 0 & 1 \\ 
5 & 3 & 6 & 6 & 4 & 6 & 0 & 3 & 1 & 1  & 0 & 5 & 6 & 4 & 4 & 6 & 2 & 3 &
3 & 6
    \end{tabular}
  \end{center}

  \begin{enumerate} \renewcommand{\theenumi}{\alph{enumi}}
  \item Classify the type of the variable.
  \item Summarize the previous information with a frequency table. 
  \item Find the mode of the number of controls. 
  \end{enumerate}
\end{problem}

\begin{problem}
A veterinary physician is interested in knowing  when emergencies
occur, so he has collected data from the last week. He has
classified them depending on the time in four groups: Morning, Afternoon,
Night and Nonworking days (Bank holidays and Sundays).
\begin{center}
\begin{tabular}{|*{7}{c|}}
\hline   Morning &Morning & Night &Nonworking & Night  & Afternoon & Night  \\ 	
\hline   Morning & Morning& Nonworking &  Night  &Afternoon & Afternoon&Morning \\ 	
\hline   Morning & Morning&Afternoon &Morning & Night  & Afternoon& Afternoon\\ 	
\hline   Morning & Afternoon &Nonworking &Morning & Night  & Nonworking&Morning \\ 	
\hline   Afternoon & Nonworking&Afternoon & Night  & & & \\ 	
\hline
\end{tabular}
  
\end{center}

\begin{enumerate} \renewcommand{\theenumi}{\alph{enumi}}
\item Classify the type of the variable.
\item Can you calculate the median of the variable?
\item Calculate absolute and relative frequencies. Summarize this information with a frequency table. 
\item Sketch a bar plot. 
\end{enumerate}
 
\end{problem}

\begin{problem}
 A research group has done a study of the copper levels in urine with
 a sample of 40 dogs between 1 and 5 years old and  the following
 values were measured:  

 \begin{center}
   \begin{tabular}{|*{10}{c|}}
  \hline    0.10 & 0.30 & 0.34 & 0.36 & 0.42 & 0.42 & 0.45 & 0.48 & 0.50 & 0.52 \\
\hline     0.55 & 0.58 & 0.62 & 0.63 & 0.64 & 0.65 & 0.65 & 0.66 &0.69 & 0.70 \\
\hline   0.72 & 0.73 & 0.74 & 0.74 & 0.75 & 0.76 &0.77 & 0.78 & 0.81 &0.83 \\
\hline   0.85 & 0.86 & 0.88 & 0.90 & 0.94 & 0.98 & 1.04 & 1.12 &1.16
&1.24  \\
\hline 
   \end{tabular}
 \end{center}
 \begin{enumerate} \renewcommand{\theenumi}{\alph{enumi}}
 \item What is the variable of  study? Classify this variable. 
 \item Calculate the median and the range. 
 \item Calculate the lower and upper quartile. 
  \item Calculate the 10-percentile and the 95-percentile.
  \item Is there any outlying value?
  \item Sketch a histogram and a box plot. 
 \end{enumerate}
\end{problem}


\begin{problem}
  In the records of a  zoo, the weight (in grams) of 16 gorillas one month after they were born are shown in the table 
  \begin{center}
    \begin{tabular}{|*{8}{c|}}
      \hline 4123 & 4336 & 4160 & 4165 & 4422 & 3853 & 3281 & 3990 \\
     \hline  4096 & 4166 & 3596 & 4127 & 4017 & 3769 & 4240 & 4194 \\
\hline
    \end{tabular}
  \end{center}
  \begin{enumerate}
  \item Classify the variable. 
  \item Calculate the following statistics: minimum and maximum, 10 and 90 percentiles, quartiles, median, mean, mode, range, variance, standard deviation. 
  \end{enumerate}
\end{problem}