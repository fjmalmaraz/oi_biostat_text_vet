\section{Exercises}
\label{ChiExercises}




\begin{problem}
  On a farm that was using a determined chemical product for
  disinfection was detected that some workers started to ace some
  disorders of the respiratory system. The disinfectant was thought to
  be related to  these disorders.  500 workers were selected to
  evaluate this hypothesis, being classified according to the level of
  exposure  and whether or not the symptoms of these disorders. The
  results are presented 
  \begin{center}
  \begin{tabular}{|c|*{3}{c}|}
    \hline               &  Direct contact  & Limited contact & No contact  \\
    \hline  Disorder symptoms &  185 & 33 & 17 \\
   Asymptomatic & 120 & 73 & 72 \\
\hline   
  \end{tabular}  
  \end{center}
  
Do we have enough evidence to state that there exists a relation
between the level of exposure and the presence of these symptoms among workers? Set and solve an adequate hypothesis testing, explain your conclusion from the study. 
\end{problem}


  \begin{problem}
   
A study at a horse  have the objective of a relation between body size of the horse  and the presence of  certain anomalies close to  knees called ``angular deformities''. to check this hypothesis, 500 horses were randomly selected and classified according to their body size within four categories and the presence of angular deformities. The outcome of this study was the following table: 

\begin{center}
\begin{tabular}{|c|*{4}{c}|}
\hline                                                     &  Low &
                                                    Medium--low & 
                                    Medium--high & 
High \\ 
 \hline  Presence of angular deformities & 8 & 24 & 32 & 27 \\
Absence of angular deformities & 42 & 121 & 138 & 108 \\ 
\hline 
\end{tabular}
  
\end{center}

Are these data compatible with the hypothesis that the presence of angular deformities is related to the body size of the horse? Use a significance level $\alpha=0.05$. Set an adequate hypothesis testing to answer this question and draw your conclusions.   
  \end{problem}


  \begin{problem}
    
A study is being performed to compare two drugs for the acute pain
provoke by bone fracture in horses.  They have been labeled as  D1 and
D2. After treatment, every animal is classified within three
categories: \emph{pain removed}, \emph{ reduced intensity} and
\emph{changes not appreciated}. The drug D1 was given to 32 horses
with bone fracture and 28 horses received drug D2 with the sampe
situation. In 12 cases the pain was removed, being 7 horses with D1
treatment and others with treatment D2. The pain was reduced for 30
horses, 17 out of 30 had a intake of D1 and the remaining horses
D2. Finally, from 18 horses where no changes where appreciated, 8 of
them have received drug D1. Could we say that one drug is more
effective than the other one to reduce the pain from bone fracture? 
Set an adequate hypothesis testing with significance level
$\alpha=0.05$ to answer this question and draw your conclusions.   
  \end{problem}

  \begin{problem}
    It is well-known that the exposure of tobacco smoke is totally harmful to the health, destroying cells in the body. One of the possible effects is thought to be  a progressive muscle atrophy. With the objective of confirming that smoking affects the mass of the muscle, an experiment was performed with lab rat. One group was exposed to tobacco smoke and the other group not.  The muscle protein synthesis   was measured  by mean of the protein content in blood and the participants were classified depending on the values was between the 10 and 90-percentile, lower than 10-percentile or higher than 90-percentile of the population.
    \begin{center}
      \begin{tabular}{|c|*{3}{c}|}
 \hline        & $<$ 10-perc. & Betw. 10 and 90-perc. &       $>$ 90-perc. \\ 
 \hline rats exposed to tobacco smoke & 117 & 529 & 19 \\
  rats not exposed to tobacco smoke & 124 & 1147 & 117 \\
\hline  
     \end{tabular}
    \end{center}
Is there enough evidence in favour of an association of the protein content and smoking? Set an adequate hypothesis testing using $\alpha=0.1$, calculate test statistics, acceptance interval and draw your conclusions.   
  \end{problem}
\begin{problem} % (From \cite{Kim}) % Ex.1 pag. 198
 At the beginning of the last flu season, Orange County Community
 Hospital staff administered a flu vaccine. To determine if there is
 an association between the flu vaccine and contraction of the flu,
 235 of the inoculated patients and 188 uninoculated subjects were
 randomly sampled. The investigators monitored the study subjects
 throughout the flu season and gathered the data shown in the table.\footfullcite{kim2008biostatistics}

 \begin{center}
   \begin{tabular}{r|cc|l}
      & \multicolumn{2}{c}{Contracted Flu} & \\
\cline{2-3} Flu vaccine & Yes & No & \\
\hline Yes & 43 & 192 & 235 \\
No & 51 & 137 & 188 \\
\hline & 94 & 329 & 423 
   \end{tabular}
 \end{center}
\begin{enumerate}
\item State the appropriate null hypothesis to test for an association between two variables.
\item  Find the expected frequency for each cell.
\item  Find the value of the test statistic.
%\item  What is the p value?
\item  State the conclusion of the significance test.
\end{enumerate}
\end{problem}
