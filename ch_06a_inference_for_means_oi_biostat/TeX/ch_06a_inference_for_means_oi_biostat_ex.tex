\section{Exercises}
\label{TwoSampleExercises}
  
\begin{problem}
  A study is performed to compare the prevalence of the Iron Deficiency Anemia in two breeds of swine: Landrace and Spotted Poland. A random sample of 450  Landrace swines were selected and 105 showed positive indicators of IDA. Meanwhile, 120 out  of 375 spotted Poland pigs have  positive indicators of IDA. Has this study got enough evidence to point out that there exists a difference of the IDA prevalence in both groups? (Use $\alpha=0.05$)
\end{problem}

\begin{problem}
  Two physiotherapeutic treatments are being tested   to relieve some
  rheumatologic symptoms in horses. The first treatment ($T_1$) has
  been tested in 150 horses and 87 of them improved their conditions after one month of treatment. In the  group receiving the second treatment $T_2$ 90 out of 170 horses have improved their conditions. Is there any difference in the efficacy of the treatments? Calculate the p-value.
\end{problem}


\begin{problem}
  A group of researchers suppose that a relevant factor in hay fever
  in horses is the climate. They design an experiment with two type of climate zones (one region with temperate climate and a cold region) in the same country. In a sample of 1350 horses in the warm zone, 95 of them were positive in the hay fever test.  In the other region, a sample of 2010 horses was selected and 113 in the sample were positive in the allergy test. Does this experiment provide enough evidence that the prevalence of hay fever is not the same in both region?   Calculate the p-value.  
\end{problem}

\begin{problem}
  A cardiovascular disease study for adult dogs was performed to
  confirm whether there exists any relation between pollution  and
  this type of diseases. A sample of 1350 dogs was selected from
  several industrial areas and 95 of them have some cardiovascular
  disorder. Meanwhile, 113 dogs in a sample of 2010 dogs living in
  unpolluted area shown cardiovascular diseases. Can this study get
  the conclusion that the polluted areas have a different proportion
  of cardiovascular diseases than the unpolluted areas? ($\alpha=0.05$) Calculate the p-value.  
\end{problem}

\begin{problem}
 The height of the pelvis is measured for gorillas. In a sample of 12
 male gorillas the mean was 13.21\,cm and the standard deviation was
 1.05\,cm. In a sample of 9 female gorillas the mean was 11.00\,cm and
 the sample standard deviation was 1.01\,cm.
Assuming that the height of the pelvis is normally distributed, answer the following questions: 
 \begin{enumerate}
 \item   Is there any difference in the variance of the pelvis height between female and male gorillas? 
 \item Is there any difference between the means of the pelvis height between female and male gorillas?
 \end{enumerate}
\end{problem}

\begin{problem} % (From \cite{Kim}) 

  %\noindent\begin{tabular}{@{}cc}
    %\begin{minipage}[b]{0.6\textwidth}
\begin{multicols}{2}     
 A 4--week weight control program was developed by a team of nutrition
scientists. To evaluate the efficacy of the program the
investigators selected 8 subjects who have body mass index higher
than 30. Subjects were given specific instructions to comply in
order to control the variables that may affect their weight, such as
physical exercise and snacks. The weight of each subject at baseline
and at the end of the 4--week clinical trial was measured. Suppose the
weight distribution is known to be normal. State and test the
hypothesis to determine if the program is
effective. \footfullcite{kim2008biostatistics}
%\columnbreak
  %  \end{minipage}&
%\begin{minipage}[b]{0.35\textwidth}

\hfill       \begin{tabular}[b]{|c|c|c|}
\hline  Subject & Baseline & Program \\ \hline
1 & 82.3 & 76.5 \\
2 & 76.5 & 73.8 \\
3 &  103.7 & 98.2 \\
4 & 96.8 & 95.8  \\
5 & 108.5 & 112.6  \\
6 & 94.3 & 89.9 \\
7 &115.7 & 111.4 \\
8 & 125.1 &  117.4 \\ \hline 
\end{tabular} \hfill \hfill
  %  \end{minipage}
  %\end{tabular}
\end{multicols}


\end{problem}

\begin{problem}  %(From \cite{Kim}) 
Social phobia has in recent years been recognized as a considerable
public health concern. It was speculated by psychiatrists that
socially phobic patients who are diagnosed as having chronic
depression may have greater fear of social interaction than those who
do not have chronic depression. Liebowitz social anxiety test was
given to 16 socially phobic subjects who suffer from chronic
depression (group I), and 21 socially phobic subjects who are not
chronically depressed (group II). The investigators computed
descriptive statistics from the social anxiety test scores: $\overline{X}_1 =
22.6$, $S_1^2 = 14.0$, $\overline{X}_2 = 20.1$, and $S_2^2 = 12.2$. If the
distribution of Liebowitz social anxiety test scores are normally
distributed,  what can you conclude from the data?
\begin{enumerate}
\item Check whether it can be assumed that the variances in the two
  groups are identical.
\item State the appropriate hypothesis according to the objectives of
  this study. 
\item  Perform the test by using the $p$ value.\footnote{Ibidem}
\end{enumerate}
\end{problem}
\begin{problem} % (From \cite{Kim}) 
It has been suggested that smoking does not affect the risk of
cardiovascular diseases in populations with low serum cholesterol
levels. To determine whether cigarette smoking is an independent risk
factor among men with low levels of serum cholesterol, a nationwide,
multicentered study was conducted. At one of the study sites, Orange
Crest Community Hospital, 25 smokers and 47 non--smokers signed the
consent form to participate in the study. Their serum cholesterol
measurements are summarized below. Suppose the distribution of serum
cholesterol levels is approximately normal. 
 What can you conclude from the  data collected by the researchers?\footnote{Ibidem}
 \begin{center}
\begin{tabular}{|l|l|l|}
\hline   Serum Cholesterol & Non--Smokers & Current Smokers \\
\hline  Sample mean & $\overline{X}_1=209.1$ & $ \overline{X}_2=213.3$ \\
Sample SD & $S_1=35.5$ & $S_2=37.6$ \\
\hline  
\end{tabular}   
 \end{center}
\end{problem}
